\newHymn

%│ This hymn was written by Jean Tisserand, O.F.M. (d. 1494) and originally had only nine stanzas. Stanzas │
%          │ "Discipulis adstantibus", "Ut intellexit Didymus", "Beati qui non viderunt" are early additions to the  │
%          │ hymn. There are several different versions of the hymn. The one below is one of the more common         │
%          │ versions.                                                                                               │
%          └─────────────────────────────────────────────────────────────────────────────────────────────────────────┘

\FirstLine{O filii et filiae}

\settowidth{\versewidth}{Rex caelestis, Rex gloriae}

\begin{verse}[\versewidth]

\FirstVerse{A}{lleluia,} Allelúia, Allelúia.

\flagverse{1}O fílii et fíliae,\\*
Rex caeléstis, Rex glóriae \\*
Morte surréxit hódie.                           
Allelúia.
\pointtrans

\flagverse{2}Ex mane prima Sábbati \\*
Ad óstium monuménti   \\*  
Accessérunt discípuli.  Allelúia.             

\flagverse{3}Et María Magdaléne,     \\*
Et Jacóbi, et Salóme    \\*
Venérunt corpus úngere  Allelúia.             

\flagverse{4}In albis sedens ángelus   \\*
Praedíxit muliéribus:     \\*
In Galilaéa est Dóminus.  Allelúia. 

\flagverse{5}Et Joánnes apóstolus \\*
Cucúrrit Petro cítius, \\*
Monuménto venit prius. 
                      Allelúia.

\flagverse{6}Discípulis adstántibus,\\*
In medio stetit Christus,\\*
Dicens: Pax vobis ómnibus. Allelúia. 

\flagverse{7}Ut intelléxit Dídymus\\*
Quia surréxerat Jesus,    \\*
Remánsit fere dúbius.     Allelúia. 

\flagverse{8}Vide Thoma, vide latus,\\*
Vide pedes, vide manus,   \\*
Noli esse incrédulus.     Allelúia. 

\flagverse{9}Quando Thomas Christi latus,\\*%vidit Christum,\\*
Pedes vidit atque manus,\\*%, manus, latus suum,         \\*
Dixit: Tu es Deus meus.           
                                 Allelúia.

\flagverse{10}Beáti qui non vidérunt \\*  
Et fírmiter credidérunt; \\*
Vitam aetérnam habébunt. 
                        Allelúia.  

\flagverse{11}In hoc festo sanctíssimo\\*
Sit laus et jubilátio:   \\*
\textsc{Benedicamus Domino.}
                        Allelúia.  

\flagverse{12}Ex quibus nos humíllimas\\*
Devótas atque débitas    \\*
\textsc{Deo} dicámus \textsc{gratias.}
                        Allelúia.

\end{verse}


%    Latin from March, Latin Hymns. Translation by 
%\Htrans{Fr. Edward Caswall}{1814--78}

\Hpoet{Jean Tisserand, O.F.M.}{d.~1494}
