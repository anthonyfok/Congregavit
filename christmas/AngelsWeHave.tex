\newHymn
\FirstLine{Angels we have heard on high}

\begin{verse}[\versewidth]
\indentpattern{22}
\begin{altverse}
\FirstVerse{A}{ngels} we have heard on high,\\*
Sweetly singing o'er the plains,\\
And the mountains in reply\\*
Echoing their joyous strains.
\end{altverse}

\begin{indentedVerse}
\flagverse{Refrain} \vin Gloria, in excelsis Deo!\\*
\vin Gloria, in excelsis Deo!
\end{indentedVerse}

\begin{altverse}
\flagverse{2} Shepherds, why this jubilee?\\*
Why your joyous strains prolong?\\
Say what may the tidings be\\*
%What the gladsome tidings be\\*
Which inspire your heavenly song.
\end{altverse}

\begin{altverse}
\flagverse{3} Come to Bethlehem and see\\*
Him whose birth the angels sing;\\
Come, adore on bended knee,\\*
Christ the Lord, the newborn King.
\end{altverse}

\begin{altverse}
\flagverse{4} See within a manger laid,\\*
%Whom the choirs of angels praise;\\
Jesus, Lord of heaven and earth!
Mary, Joseph, lend your aid,\\*
With us sing our Saviour's birth.
%While our hearts in love we raise.
\end{altverse}

\end{verse}

\Htrans{Bishop James Chadwick}{1813--82}


%                                Words: Traditional French carol, "Les Anges dans nos Campagnes"
%                            Translated from French to English by Bishop James Chadwick (1813-1882);
%          Appeared in Holy Family Hymns (1860) and The Crown of Jesus Music (1864, adapted by Henri Friedrich Hémy).
