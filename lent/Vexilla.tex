\newHymn
\FirstLine{Vexilla Regis prodeunt}

\begin{multicols}{2}
\begin{verse}[\versewidth]

\FirstVerse{V}{exilla} Regis pródeunt;\\*
Fulget Crucis mystérium,         \\
Qua vita mortem pértulit,\\*
Et morte vitam prótulit.
%Quo carne carnis conditor           \\*        
%Suspensus est patibulo.
\pointtrans

%\flagverse{2}Confixa clavis viscera                 \\*     
%tendens manus, vestigia,                  \\*  
%redemptionis gratia                         \\*
%hic immolata est hostia.                    

\flagverse{2}Quae vulneráta lánceae                      \\*
Mucróne diro críminum         \\
Ut nos laváret sórdibus,                   \\*
Manávit und\emph{a} et sánguine.                   

\flagverse{3}Impléta sunt quae cóncinit                  \\*
David fidéli cármine,                       \\
Dicéndo natiónibus:                         \\*
Regnávit a ligno Deus.                      

\flagverse{4}Arbor decór\emph{a} et fúlgida,                    \\*
Ornáta Regis púrpura,                       \\
Electa digno stípite              \\*
Tam sancta membra tángere.                  

\flagverse{5}Beáta, cujus bráchiis                       \\*
Prét\emph{i}um pepéndit saéculi:                   \\
Statéra facta córporis,                     \\*
Tulítque praedam tártari.

%\flagverse{7}Fundis aroma cortice,                       \\*
%vincis sapore nectare,                      \\*
%iucunda fructu fertili                      \\*
%plaudis triumpho nobili.                    

%\flagverse{8}Salve, ara, salve, victima,                 \\*
%de passionis gloria,                        \\*
%qua vita mortem pertulit                    \\*
%et morte vitam reddidit.                    

\flagverse{6}O Crux ave, spes única,                     \\*
Hoc Passiónis témpore.                   \\
Piis adáuge grátiam,                        \\*
Reísque dele crímina.                       

\flagverse{7}Te, fons salútis Trínitas,                  \\*
Colláudet omnis spíritus:                   \\
Quibus Crucis victóriam\\*
Largíris, adde praémium.\\
Amen.
%Quos per Crucis mysterium                   \\*
%Salvas, fove per saecula. Amen.

\end{verse}
\end{multicols}
\Hpoet{Venantius Fortunatus}{530--609}
