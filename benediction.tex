\def\tabmark{Benediction}

\chapter{Benediction of the Blessed Sacrament}
%\thispa{plain}

\label{benediction}

%\section{At Exposition}

\rubrics{At the moment of exposition, an anthem or hymn to the Blessed Sacrament is sung: \emph{O Salutaris} or another one.}

\newHymn

\FirstLine{O salutaris Hostia}

\begin{verse}[\versewidth]

\FirstVerse{O}{ salutaris} Hóstia\\*
Quae coeli pandis óstium\\*
Bella premunt hostília\\*
Da robur fer auxílium.

Uni trinóque Dómino\\*
Sit sempitérna glória\\*
Qui vitam sine término\\*
Nobis donet in pátria.

\end{verse}

\fortranC{the last two verses of \ref{hymn:OSavingVictim}}

\newHymn

\HymnTitleit{O salutaris (Verbum supernum)}

\gresetfirstlineaboveinitial{8}{8}
\includescore{gabc/o_salutaris.tex}

\newpage

\heading{Prayer for the Conversion of Australia}

\noindent Let us pray,

\lettrine{O}{~ God,} Who hast appointed Mary, Help of Christians, 
St Francis Xavier and St Thérèse of the Infant Jesus, 
Patrons of Australia, grant that through their intercession 
our brethren outside the Church may receive the light of faith, 
so that Australia may become one in faith under one shepherd.  
Through Christ our Lord.  \Rbar Amen.

Mary, Help of Christians, \Rbar pray for us.

St Francis Xavier, \Rbar pray for us.

St Thérèse of the Infant Jesus, \Rbar pray for us. 

St Mary of the Cross, \Rbar pray for us. 


%\newpage

%\input localprayers.tex

\rubrics{A time of adoration follows.}

%\newpage

%\section{At Benediction}

\rubrics{Before the blessing (the Benediction, properly so called) the \emph{Tantum ergo} is always sung. A low bow is made at: \emph{Veneremur cernui}.}

\newHymn

\FirstLine{Tantum ergo Sacramentum}

\begin{verse}[\versewidth]
\begin{altverse}

\FirstVerse{T}{antum} ergo Sacraméntum\\*
Venerémur cérnui;\\*
Et antíquum documéntum\\*
Novo cedat rítui:\\*
Praestet fides suppleméntum\\*
Sénsuum deféctui.
\end{altverse}

\begin{altverse}
Genitóri, Genitóque\\*
Laus et jubilátio:\\*
Salus, honor, virtus quoque\\*
Sit et benedíctio:\\*
Procedénti ab utróque\\*
Compar sit laudátio.  Amen.

\end{altverse}
\end{verse}

\fortranC{the last two verses of \ref{hymn:DownInAdoration}}

\newHymn

\HymnTitleit{Tantum ergo (Pange lingua)}

\gresetfirstlineaboveinitial{3}{3}
\includescore{gabc/tantum_ergo.tex}

\newHymn

\HymnTitleit{Tantum ergo (Spanish Chant)}

\gresetfirstlineaboveinitial{5}{5}
\includescore{gabc/tantum_ergo_spanish.tex}

\bigskip

\paracoltest

\begin{Parallel}{\gcolwidth}{\gcolwidth}
%\ParallelLText
\latin{\Vbar Panem de coelo prae\-sti\-tísti eis. (T.~P.\ Alleluia)}
\vern{Thou hast given them bread from heaven. (P.~T.~Alleluia)}
\latin{\Rbar Omne delectaméntum in se habéntum. (T.~P.\ Alleluia)}
\vern{Having in itself all delight. \\(P.~T.~Alleluia)}
\latin{Orémus}
\vern{Let us pray.}
\firstlatin{D}{eus,}{ qui pro nobis sub Sacraménto mirábili passiónis tuae memóriam \parfillskip0pt
re\-li\-quísti :}
\firstvern{O}{~ God,}{ Who, under a wonderful Sacrament, hast left us a memorial of \parfillskip0pt
Thy}
\latin{tríbue quaésumus, ita nos córporis et sánguinis tui sacra
mystéria venerári, ut redemptiónis tuae fructum in nobis júgiter
sentiámus.  Qui vivis et regnas in saecula saeculorum. \Rbar Amen.}
\vern{Passion;
grant us, we beseech Thee, so to venerate the sacred mysteries of Thy
Body and Blood, that we may ever feel within us the fruit of Thy redemption.
Who livest and reignest, world without end. Amen.}
\end{Parallel}

%\newpage

\heading{The Divine Praises}

Blessed be God.\\
Blessed be His Holy Name.\\
Blessed be Jesus Christ, true God and true man.\\
Blessed be the name of Jesus.\\
Blessed be His Most Sacred Heart.\\
Blessed be His Most Precious Blood.\\
Blessed be Jesus in the most Holy Sacrament of the Altar.\\
Blessed be the Holy Spirit, the Paraclete.\\
Blessed be the great Mother of God, Mary most holy.\\
Blessed be her holy and Immaculate Conception.\\
Blessed be her glorious Assumption.\\
Blessed be the name of Mary, Virgin and Mother.\\
Blessed be Saint Joseph, her most chaste spouse.\\
Blessed be God in His angels and in His saints.

\bigskip

\rubrics{The service may be concluded by the following Psalm \emph{Laudate Dominum} (with or without the Antiphon \emph{Adoremus}), or another suitable hymn.}

\newHymn

\HymnTitleit{Adoremus in aeternum}

\smallskip

\paracoltest

\begin{Parallel}{\gcolwidth}{\gcolwidth}
%\ParallelLText{%\begin{Parallel}{\gcolwidth}{\gcolwidth}
%\noindent\begin{minipage}[t]{0.48\columnwidth}%
\firstlatin{A}{dorémus}{ in aetérnum sanctíssimum sacraméntum.}
\firstvern{L}{et}{ us adore forever the most holy Sacrament.}
\latin{Laudáte Dóminum omnes gen\-tes: * laudáte eum omnes pópuli.}
\vern{Praise the Lord all you nations, praise Him all you peoples.}
\latin{Quóniam confirmáta est super nos, misericórdia eius: * et véritas Dómini
manet in aetérnum.}
\vern{For His Mercy is confirmed upon us, and the truth of the Lord endures eternally.}
\latin{Glória Patri et Fílio, * et Spirítui Sancto.}
\vern{Glory be to the Father and to the Son and to the Holy Spirit.}
\latin{Sicut erat in princípio, et nunc et semper, * et in saécula saeculórum.
Amen.}
\vern{As it was in the beginning, is now and ever shall be, world without
end. Amen.}
\latin{Adorémus in aetérnam sanctíssimum sacraméntum.}
\vern{Let us adore forever the most holy Sacrament.}

\end{Parallel}

%\newpage

\newHymn

\HymnTitleit{Adoremus in aeternam (Chant)}

\includescore{gabc/adoremus.tex}

\newHymn

\HymnTitleit{Cor Jesu sacratissimum}

\includescore{gabc/corJesu.tex}

\newHymn

\HymnTitleit{Laudemus Dominum}

\includescore{gabc/laudemusDominum.tex}



